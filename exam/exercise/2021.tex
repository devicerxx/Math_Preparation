\newtheorem*{remark}{\color{orange}{注}}

\title{复旦大学2021每周一题}
\maketitle

\section{高代 \uppercase\expandafter{\romannumeral 1} 每周一题}

\begin{groups}
    \begin{questions}
        \question xxx
    \end{questions}

    \begin{solution}{4cm}
        \method 橘子数学社区的宗旨是开放、高效、合作、变革.
        \qed
    \end{solution}
\end{groups}

\setcounter{Question}{0}

\title{复旦大学2022每周一题}
\maketitle

\section{高代 \uppercase\expandafter{\romannumeral 2} 每周一题}

\begin{groups}
    \begin{questions}
        \question xxx
    \end{questions}

    \begin{questions}
        \question 
        设$A,B$分别是数域$\mathbb{K}$上的$m,n$阶矩阵,他们在复数域$\mathbb{C}$中有
        公共的特征值,证明:存在非零矩阵$C \in M_{m \times n}(\mathbb{K})$,使得
        $AC=CB$.
    \end{questions}

    \begin{questions}
        \question 
        设$A$是数域$\mathbb{K}$上的$n$阶矩阵,其特征多项式等于极小多项式,证明:
        矩阵方程$XA=\ts{A}X$的解是$\mathbb{K} $上的对称阵.
    \end{questions}

    \begin{questions}
        \question 
        设$V$为$n$阶复方阵全体构成的线性空间,$V$上的线性变换$\varphi$定义为$\varphi(X)=AX-XA$,
        其中$A \in V$. 证明:$\varphi$可对角化的充要条件是$A$可对角化.
    \end{questions}

    \begin{remark}
        本题是第二届CMC决赛题的推广. 
    \end{remark}

    \begin{questions}
        \question 
        设$V$为$n$阶复方阵全体构成的线性空间,$V$上的线性变换$\varphi$定义为$\varphi(X)=JXJ$,
        其中$J=J_n(0)$是特征值为$0$的$n$阶Jordan块,试求$\varphi$的Jordan标准型.
    \end{questions}

    \begin{questions}
        \question 
        设$a$为实数,求下列$n$阶实对称阵的正负惯性指数:
        $$
            A = \begin{pmatrix}
                1 & a & a^2 & \cdots & a^{n-1} \\
                a & 1 & a & \cdots & a^{n-2} \\
                a^2 & a & 1 & \cdots & a^{n-3} \\
                \vdots & \vdots & \vdots & & \vdots \\
                a^{n-1} & a^{n-2} & a^{n-3} & \cdots & 1
            \end{pmatrix}
        $$
    \end{questions}

    \begin{questions}
        \question 
        设$A=(a_{ij})$为$n$阶正定实对称阵,$B=(b_{ij})$为$n$阶半正定实对称阵且主对角元全打于零,证明:
        Hadamard乘积$A\circ B = (a_{ij} b_{ij})$是正定对称阵.
    \end{questions}

    \begin{questions}
        \question 
        设$A$为$n$阶实对称阵,$B$为$n$阶半正定实对称阵,满足$|A+iB|=0$,求证:存在$n$维非零
        实列向量,使得$A\alpha = B \alpha = 0$.
    \end{questions}
    
    \begin{remark}
        FDU21期末考试第八题推广.
    \end{remark}

    
\end{groups}