% 使用 BHCexam 文档类,并传递选项
\documentclass{BHCexam}
\usepackage{hyperref}
\usepackage[normal]{fixdif}

\begin{document}
	
	% 第一行主标题
	\title{ZJU 2021 高等代数}
	% \notice{注:满分150分, 180分钟完成.}
	
	% 生成试卷头
	\maketitle
	
	\begin{groups}
        \begin{questions}
            \question 试求$t$的值,使得多项式$f(x)=x^3+6x^2+tx+8$具有重根,并求出相应的重根.
        \end{questions}

        \begin{questions}
            \question[15] 
            已知可逆方程$A$的逆为$A^{-1}=\begin{pmatrix}
                1 & 1 & 1 & 1\\
                -1 & 2 & 1 & -2 \\
                1 & 4 & 1 & 4 \\
                -1 & 8 & 1 & -8
            \end{pmatrix}$,求$\sum \limits_{i=1}^4 \sum \limits_{j=1}^4 i A_{ij}$.
        \end{questions}

        \begin{questions}
            \question[15] 
            设$a_1,\dots,a_s$为线性方程组$Ax=0$的一组基础解系,另有一组向量$a$.
        \end{questions}

        \begin{questions}
            \question[15]
            若$n$阶复矩阵$A,B$满足$AB=BA$,且$B$有$n$个不同的特征值,证明:$A$可对角化.
        \end{questions}

        \begin{questions}
            \question[15]
            设$c_1,c_2,c_3$实多项式$f(x)=2x^3-4x^2+6x-1$的三个复根,求$(c_1c_2+c_3^2)(
                c_2c_3+c_1^2
            )(c_1c_3+c_2^2)$.
        \end{questions}

        \begin{questions}
            \question[20]
            在$R^2$上$f(x,y)=a_{11}x^2+2a_{12}xy+a_{22}y^2+2b_1x+2b_2y+c$,令$A_f=\begin{pmatrix}
                a_{11} & a_{12} \\
                a_{12} & a_{22}
            \end{pmatrix}$. $B_f=\begin{pmatrix}
                a_{11} & a_{12} & b_1 \\
                a_{12} & a_{22} & b_2 \\
                b_{1} & b_{2} & c \\
            \end{pmatrix}$,证明:函数$f(x,y)$在坐标变换$\begin{pmatrix}
                x^\prime \\
                y ^\prime
            \end{pmatrix}= Q \begin{pmatrix}
                x \\ y
            \end{pmatrix}+ \begin{pmatrix}
                d_1 \\ d_2
            \end{pmatrix}下$,$tr(A_f),det(A_f),def(B_f)$保持不变,其中$Q$是二阶正交矩阵.
        \end{questions}

        \begin{questions}
            \question[20]
            设实矩阵$A=\begin{pmatrix}
                a & b \\ c & d
            \end{pmatrix},a,b,c,d>0$,证明:一定存在$A$的特征向量$\begin{pmatrix}
                x \\ y
            \end{pmatrix} \in R^2$,其中$x,y>0$.
        \end{questions}

        \begin{questions}
            \question[15]
            设$6$阶复矩阵$A,B$是幂零矩阵,且有相同的秩和最小多项式,证明:$A,B$相似.
        \end{questions}

        \begin{questions}
            \question[20]
            设$A$是$n$阶实矩阵,$B$是$n$阶正定矩阵.
            \begin{subquestions}
                \subquestion 证明:存在唯一$n$阶实矩阵$C$满足$BC+CB=A$.
                \subquestion 证明:对$(1)$中实矩阵$C$有$BC=CB$当且仅当$AB=BA$.
            \end{subquestions}
        \end{questions}
	\end{groups}
\end{document}