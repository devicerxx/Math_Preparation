% 使用 BHCexam 文档类,并传递选项
\documentclass{BHCexam}[windows]
\usepackage{hyperref}
% \usepackage[normal]{fixdif}
% \usepackage{ctex, draftwatermark, everypage}
% \SetWatermarkText{文字水印}
% \SetWatermarkLightness{0.95}
% \SetWatermarkScale{0.8}
\usepackage{unicode-math}
\usepackage{amsmath,amssymb,mathrsfs}

\usepackage{mathtools} % added
\usepackage{calc}
\usepackage{mathdots}
\usepackage{bbm}

\let\mathbb\relax % remove the definition by unicode-math
\DeclareMathAlphabet{\mathbb}{U}{msb}{m}{n}

\newcommand{\getlengthratio}[2]{%
    \number\numexpr
    \dimexpr#1\relax
    /
    \dimexpr#2\relax
    \relax} 

\newcommand{\GetMatrix}{}
\newlength{\MWi}
\newlength{\MWii}
\newcounter{MWiii}

\newcommand{\squarefy}[2]{%
\renewcommand{\GetMatrix}{} 
\renewcommand{\arraystretch}{1}
\renewcommand{\GetMatrix}{#1}
\setlength{\MWi}{\widthof{$\displaystyle\GetMatrix$}}
\setlength{\MWii}{\heightof{$\displaystyle\GetMatrix$}}
\setcounter{MWiii}{\getlengthratio{0.5\MWi}{\MWii}}
\renewcommand{\arraystretch}{\theMWiii} %% added
\[#2#1\]}

\begin{document}
	
	% 第一行主标题
	\title{ECNU 2022 高等代数}
	% \notice{注:满分150分, 180分钟完成.}
	
	% 生成试卷头
	\maketitle
	
	\begin{groups}
        \begin{questions}
            \question
            考虑数域$\mathbb{K}$上的线性方程组
            \begin{equation*}
                \left\{ 
                \begin{aligned}
                    &x_1 + x_2 + 2a x_3 = 2 \\
                    &x_1 + 3b x_2 + x_3= 2 \\
                    &x_1 + x_2 - a x_3 = 1
                \end{aligned}
                \right.
            \end{equation*}
            问$a,b$取何值时,方程组无解,有唯一解,有无穷多解,且在方程组有解时,求出所有解.
        \end{questions}

        \begin{questions}
            \question
            设3阶实对称阵$A$的秩为$2$,且$-2$是它的二重特征值,若$(1,0,0)^\prime, (2,1,1)^\prime$都是
            $A$的属于特征值$-2$的特征向量,求矩阵$A$.
        \end{questions}

        \begin{questions}
            \question 
            考虑未定元为$x,y$的次数至多为2的复系数二元多项式空间,求线性变换$\mathscr{A}: \, f(x,y) \mapsto 
            f(2x+1,2y+1)$的Jordan标准型.
        \end{questions}

        \begin{questions}
            \question
            设$\sigma$是有限维欧氏空间$V$上的正交变换,且满足$\sigma^m=\epsilon$,这里$m \geq 1, \epsilon$为恒等变换,
            记$V^\sigma = \{ v \in V: \, \sigma(v) = v \},V^\sigma$的正交补记为$V^{\sigma \perp}$.
            \begin{subquestions}
                \subquestion 对于$v \in V$,定义$\bar{v}=\dfrac{1}{m} \sum \limits_{i=1}^m \sigma^i(v),$证明:$\bar{v} \in V^\sigma$.
                \subquestion 证明:若将$v\in V$展开成$v = v_1 + v_2$,其中$v_1 \in V^\sigma, v_2\in V^{\sigma \perp}$,
                则$v_1 = \bar{v}$.
            \end{subquestions}
        \end{questions}

        \begin{questions}
            \question
            设$f(x)$是次数大于0的整系数多项式,若$2-\sqrt{3}$是$f(x)$的根,证明:
            $2+\sqrt{3}$也是$f(x)$的根.
        \end{questions}

        \begin{questions}
            \question
            设$V$是在复数域$\mathbb{C}$上的$n$维线性空间,$\mathscr{A}$是$V$上的线性变换.
            \begin{subquestions}
                \subquestion 证明:存在正整数$k\leq n$,使得$Im \, \mathscr{A}^k = Im \, \mathscr{A}^{k+1} = \cdots = Im \, \mathscr{A}^n$且
                $Ker \, \mathscr{A}^k = Ker \, \mathscr{A}^{k+1}  =\cdots =  Ker \, \mathscr{A}^n = $
                \subquestion 考虑如下特征指标:
                \begin{enumerate}
                    \item $\mathscr{A}$的秩为$r$.
                    \item $\mathscr{A}$的特征值为0的Jordan块个数为$m$.
                    \item $\mathscr{A}$的特征值为0的Jordan块阶数为$n$.
                    \item 第(1)问种出现的最小的$k$.
                \end{enumerate}
            \end{subquestions}
        \end{questions}

        \begin{questions}
            \question
            设$V$是实内积空间,$\langle \cdot \rangle$是$V$上的内积,$\varphi$是$V$上的可逆线性变换满足:
            $$
                \langle \varphi \left(\varphi(x) \right) ,y \rangle = \langle x, \varphi(y)
                \rangle, \, \forall x,y \in V. 
            $$
            证明:$\varphi$是正交变换.
        \end{questions}

        \begin{questions}
            \question
            设$U,V,W$是6维空间的3个3维子空间,设$U \cap V = 0$,求$dim\,[(U+V) \cap (V+W)]$的
            最大值和最小值.
        \end{questions}

        \begin{questions}
            \question
            \begin{subquestions}
                \subquestion 设$A\in M_n(\mathbb{R})$是半正定实对称阵,$x\in \mathbb{R}^n$,证明:$x^\prime A x = 0$等价于$Ax=0$.
                \subquestion 设$A$为$n$阶半正定实对称阵,将其写成分块矩阵的形式
                \squarefy{\begin{pmatrix}
                    A_1 & A_2 \\
                    A_2^\prime & A_4
                \end{pmatrix}}{A = }
                其中$A_1$是$r$阶方阵,证明对$x\in \mathbb{R}^r$,若$A_1x=0$,则$A_2^\prime x=0$.
                \subquestion 设$A,B$是$n$阶半正定实对称阵,且$r(A)=r$. 证明:存在$n$阶可逆阵$p$使得
                $$
                    P^\prime A P = \begin{pmatrix}
                        I_r & O \\
                        O & O
                    \end{pmatrix}, \quad P^\prime B P = diag \{ \lambda_1, \dots, \lambda_n \}
                $$
                其中$I_r$为$r$阶单位阵.
            \end{subquestions}
            

        \end{questions}
	\end{groups}
\end{document}