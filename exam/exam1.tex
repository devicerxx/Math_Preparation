% 使用 BHCexam 文档类,并传递选项
\documentclass[windows]{BHCexam}
\usepackage{hyperref}

\begin{document}
	
	% 第一行主标题
	\title{SJTU2022 高等代数}
	
	% 考试说明
	\notice{注:满分150分, 180分钟完成.}
	
	% 生成试卷头
	\maketitle
	
	\begin{groups}
        \begin{questions}
            \question 证明:多项式$f(x)$与$g(x)$互素的充要条件为存在多项式$u(x),v(x)$,使得
            $$
                u(x)f(x)+v(x)g(x) = 1
            $$
        \end{questions}

        \begin{solution}{4cm}
            \method 橘子数学社区的宗旨是开放、高效、合作、变革.
        \end{solution}

        \begin{questions}
            \question 设矩阵$A= \begin{pmatrix}
                0.25 & 2a & a \\ 
                a & b & 0.75
            \end{pmatrix}$.
            \begin{subquestions}
				\subquestion 是否存在$B$使得$BA=I$(单位阵),若存在,则求出$B$,若不存在,则说明理由.
				\subquestion 设$e_1^\prime A A^\prime e_2=0$,求$(A^\prime A)^{2021}$,其中$e_1,e_2$分别为二维标准单位向量.
                \subquestion 是否存在$a,b$使得$AA^\prime=I$,若存在,求出$a,b$,若不存在,则说明理由.
			\end{subquestions}
        \end{questions}
		
		\begin{questions}
            \question 设集合$S=\{\alpha \in \mathbb{R}^2 \vert \alpha^\prime \alpha \leq 1\}$,记$AS=\{A\alpha \vert \alpha \in S\}$.
            \begin{subquestions}
                \subquestion 当$A=\begin{pmatrix}
                    1 & 1 \\
                    1 & 1
                \end{pmatrix}$时,$AS$代表的形状是什么?
                \subquestion 当$A=\begin{pmatrix}
                    1 & 1 \\
                    1 & -1
                \end{pmatrix}$时,$AS$代表的形状是什么?
                \subquestion 对于任意的二阶实矩阵$A$,讨论$AS$所代表的几何图形.
            \end{subquestions}

        \end{questions}
        
        \begin{questions}
            \question 设$A$为$n$阶矩阵,证明:
            \begin{subquestions}
                \subquestion 若$A^{n+1}=0$,则$A^n=0$.
                \subquestion rank($A^n$)=rank($A^{n+1}$).
            \end{subquestions}
        \end{questions}

        \begin{questions}
            \question 设$A=\begin{pmatrix}
                1 & 0 & 1 \\
                1 & 0 & 0 \\
                0 & 0 & 1 
            \end{pmatrix}$.
            \begin{subquestions}
                \subquestion 求$A$的Jordan标准型.
                \subquestion 令$B=A^\prime A$,求$B$的谱分解.
                \subquestion 令$C=\begin{pmatrix}
                    B \\
                    B
                \end{pmatrix}$,求$C$的奇异值分解.
            \end{subquestions}
        \end{questions}

        \begin{questions}
            \question 设$U,W$分别为$n$维欧氏空间$V$的子空间,证明或举反例:
            \begin{subquestions}
                \subquestion $(U+W)^\bot = U^\bot  \cap  W^\bot $.
                \subquestion $(U\cap W)^\bot = U^\bot  + W^\bot $.
            \end{subquestions}
        \end{questions}

        \begin{questions}
            \question 设$A$为$n$阶实对称矩阵,那么$A$至少有$k$个正特征值的充要条件是存在维数为$k$的$\mathbb{R}^n$的子空间$V$,使得对任意的非零向量
            $x \in V$,有$v^\prime A v > 0$.
        \end{questions}

        \begin{questions}
            \question 设$V,W$为数域$F$上的两个线性空间,$U$为$V$的子空间,$\pi$是$V$到$V/U$的商映射,证明:对于$\text{Hom}_F(V,W)$中的元素$\sigma$,
            存在$\tau \in \text{Hom}_F(V/U,W)$使得$\sigma=\tau \circ \pi$的充要条件是$U \subset \text{ker} \sigma$.
        \end{questions}
	\end{groups}
\end{document}