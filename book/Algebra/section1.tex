\chapter{多项式}

\begin{introduction}
	\item 存一寸光阴,换一个世纪.
	\item -- <爱久见人心>
\end{introduction}

\section{整除}

整除于多项式的根密切挂钩,如果$g(x) \, | \, f(x)$则$g(x)$的根也是$f(x)$的根。这部分
还有着友阵,单位根等技巧的灵活运用。

\begin{proposition}
	假设$f_0(x^5) + x\cdot f_1(x^{10}) + x^2 \cdot f_2(x^{15}) + x^3 \cdot f_3(x^{20})$能被
	$x^4+x^3+x^2+1$整除,证明:$f_i(x)$能被$x-1$整除.
\end{proposition}

\begin{proof}
	这个多项式$x^4+x^3+x^2+x+1$,特别有规律,最后一项$1$乘以$x$得到第二项,第二项依次乘可以一直做下去
	,有点像循环子空间的感觉了,于是我们乘以$x$再减去他们本身,即用$x-1$去乘这个多项式:
	\begin{align*}
		&(x-1)(x^4+x^3+x^2+x+1)\\
		&=x^5 + x^4 +x^3+x^2+x - x^4 - x^3 - x^2- x - 1\\
		&=x^5-1
	\end{align*}
	即$x^4+x^3+x^2+x+1$是$x^5-1$的因式.

	设5次单位根$\omega_k = \cos \dfrac{2k \pi}{5} + i \cdot \sin \dfrac{2k \pi}{5},(k=0,1,\dots,4)$,$\omega_k^5=1$.
	\begin{equation}
		\begin{cases}
			f_0(1) + \omega_1 \cdot f_1(1) + \omega_1^2 \cdot f_2(1) + 
			\omega_1^3 \cdot f_3(1) = 0 \\
			f_0(1) + \omega_2 \cdot f_1(1) + \omega_2^2 \cdot f_2(1) + 
			\omega_2^3 \cdot f_3(1) = 0 \\
			f_0(1) + \omega_3 \cdot f_1(1) + \omega_3^2 \cdot f_2(1) + 
			\omega_3^3 \cdot f_3(1) = 0 \\
			f_0(1) + \omega_4 \cdot f_1(1) + \omega_4^2 \cdot f_2(1) + 
			\omega_4^3 \cdot f_3(1) = 0 \tag{\textasteriskcentered}
		\end{cases}
	\end{equation}
	写出$\ast $的系数行列式:
	\begin{align*}
		\begin{vmatrix}
			1 & \omega_1 & \omega_1^2 & \omega_1^3 \\
			1 & \omega_2 & \omega_2^2 & \omega_2^3 \\
			1 & \omega_3 & \omega_3^2 & \omega_3^3 \\
			1 & \omega_4 & \omega_4^2 & \omega_4^3 
		\end{vmatrix}
		=\prod _{1\leq i < j \leq 4} (\omega_j - \omega_i)\neq 0.
	\end{align*}
	$\implies$ 方程组$(\ast) $只有零解,即$f_i(1)=0,(\forall \,1\leq i \leq 4)$,即$x-1 \, | \, f_i(x)$.
	\qed{}
\end{proof}

\begin{remark}
	如果是$x^4+x^2+1$,我们观察到得乘$x^2$才会匹配,于是,乘以$x^2-1$. 暂时归结为技巧吧
\end{remark}