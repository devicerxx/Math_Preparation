
\chapter{每日一题CMC}

\begin{introduction}
	\item \emph{The summer has ended, and we are not saved.}
	\item \emph{夏天已过,我们还未得救.}
	\item \quad \quad \quad \quad --\emph{《橘子不是唯一的水果》}
\end{introduction}

总所周知,一个数学系的学生,不会做题,那岂不是丢大脸了,写不出来是一回事,没有思路是另一回事,
鉴于没有参加过CMC,又是大四阶段的数学学院学生,就得多写...ZJU除了出原创题,还特别喜欢CMC的题目.

\section*{Day 1}

\begin{ex}[CMC 1]
	设$V$是复数域$\mathbb{C}$上$n$为线性空间$(n > 0)$, $f,g$是$V$上的线性变化,如果$fg - gf = f$.
	证明:$f$的特征值都是0,且$f,g$有公共的特征向量.
\end{ex}

\begin{proof}
	\emph{Methods\,\,\uppercase\expandafter{\romannumeral 1}: 代数-几何混杂}

	随意选取一组基,则$f,g$在这组基下的表示矩阵为$A,B$,则条件可转化为$AB-BA=A$.

	要证明所有的特征值都是0,我们首先想到设出所有的特征值$\lambda_1,\lambda_2,\dots\lambda_n$. 那么
	特征值的和也是$0$,因此考虑用迹的性质.
	\begin{align*}
		tr(AB-BA) &= tr(AB) - tr(AB) = tr(A) = 0. \\
		tr(A^kB-A^{k-1}BA) &= tr(A^kB) - tr(A A^{k-1} B) = tr(A^k) = 0.
	\end{align*}
	这里我们用到了迹的性质:$tr(AB) = tr(BA)$,可是在三个及三个以上的时候我们得注意交换的次序
	$tr(ABC) = tr(CAB) = tr(BCA)$,注意到这里是错排,也就是都不能在自己原来的位置上.

	于是我们有了$tr(A)=tr(A^k)=\sum_{i=1}^n \lambda_i = \sum_{i=1}^n \lambda_i^k=0$.

	考虑Newton公式,
	\begin{align*}
		k \leq n-1&,\\
		& s_k - s_{k-1} \sigma_1 + \dots + (-1)^k k\sigma_k = 0. \\
		k \geq n \quad\,\,\,\, &,\\
		& s_k - s_{k-1} \sigma_1 + \dots + (-1)^n s_{k-n} \sigma_n = 0.
	\end{align*}
	于是:$\sigma_1=\sigma_2=\dots=\sigma_n=0$.
	
	将$\lambda_1,\lambda_2,\dots,\lambda_n$看作某个$n$次方程的根. 则
	$$
		f(\lambda) = \lambda^n - \sigma_1 \lambda^{n-1} + \cdots + (-1)^{n}\sigma_s = \lambda^s.
	$$
	则方程为$\lambda^n=0$,当且仅当$\lambda_i=0$,即所有$A$的特征值为$0$.

	$\forall \alpha \in \Ker \, f$,则$fg(\alpha) - gf(\alpha) = f (\alpha)$,得出$f\bigl( g(\alpha) \bigr) = 0$,
	即$g(\alpha) \in \Ker \, f$,所以$\Ker \, f$为$g-$不变子空间,则可以做限制,$g \lm{\Ker\,f}$一定有特征向量,因为
	是在复数域上,从而$f,g$有公共的特征向量. 

	其实这题可以用完全的几何方法,比代数简单一些,至少不用背Newton公式了是不是。

	\emph{Methods\,\,\uppercase\expandafter{\romannumeral 2}: 几何}

	任取$\lambda_0$为$f$的任一一个特征值,$V_0$为对应的特征子空间。此时$\forall \alpha \in 
	V_0$,我们都有:
	\begin{align*}
		fg (\alpha) - gf (\alpha) &= f(\alpha) \\
		f \bigl( g(\alpha) \bigr) - \lambda_0 g(\alpha) &= \lambda \alpha
	\end{align*}
	\qed{}
\end{proof}

\begin{itemize}
	\item 如果没有天赋,你还会继续你最热爱的那件事吗?
	\item 以前豆瓣给我印象很深的地方在于,豆瓣以外的地方,人们只会问你“有用吗”“电影院能看吗”,可是豆瓣
		  的友邻会说“要继续下去啊”“期待你的新作”
\end{itemize}