\chapter{对角化}\label{section3}

\begin{introduction}
	\item Schur引理
	\item Cayley-Hamilton定理
	\item Jordan-Chevalley定理
	\item Vose分解
	\item 相似标准型
	\item 极小多项式
	\item Jordan标准型
\end{introduction}

我们在线性映射里学习到,表示阵不依赖基向量的选取,因为他们都是相似的关系,那么如何找到一组特别的基向量
,让表示阵在这组基下形式特别简单。所以矩阵相似在高等代数的考试中占比特别大,是很重要的内容,仔细参考
白皮书和高等代数典型问题与方法两本书,梳理了特别重要的定理和典型例题来更好的理解这块内容.

\section{基本定理梳理}

\begin{theorem}[Schur引理]
	任意方阵必复相似于一个上三角阵.
\end{theorem}

\begin{proof}
	对矩阵的阶数$n$进行归纳,当$n=1$时,显然成立.

	假设小于$n$时成立,下面证明等于$n$时也成立.

	设$\lambda$是$A$的一个特征值,则存在非零向量$\alpha_1$,使得
	$$
		A \alpha_1 = \lambda \alpha_1
	$$
	将$\alpha_1$扩充为$\mathbb{C}_n$上的一组基,按照列分块为$P=(\alpha_1, \alpha_2, \dots, \alpha_n)$,
	因此$P$是非异阵,且
	\begin{align*}
		AP &= (A\alpha_1, A\alpha_2, \dots, A\alpha_n) \\
			& = (\alpha_1, \alpha_2, \dots, \alpha_n) \begin{pmatrix}
				\lambda & \ast  \\
				O & A_1
			\end{pmatrix}
	\end{align*}
	其中$A_1$是一个$n-1$阶方阵,注意到$P$非异,则
	$$
		P^{-1} A P = \begin{pmatrix}
			\lambda & \ast  \\
			O & A_1
		\end{pmatrix}
	$$
	由归纳假设可知,存在一个非异阵$Q$,使得
	$
		Q^{-1}A_1 Q
	$为上三角阵.
	
	令$R=\begin{pmatrix}
		1 & O \\
		O & Q
	\end{pmatrix}$,且$Q$可逆,分块对角阵的逆就是各分块的逆. 我们可以得
	\begin{align*}
		R^{-1}P^{-1} A P Q &= \begin{pmatrix}
			1 & O \\
			O & Q^{-1}
		\end{pmatrix} 
		\begin{pmatrix}
			\lambda & \ast  \\
			O & A_1
		\end{pmatrix}
		\begin{pmatrix}
			1 & O \\
			O & Q
		\end{pmatrix} \\
		&= \begin{pmatrix}
			\lambda & \ast \\
			O & Q^{-1}A_1 Q
		\end{pmatrix}
	\end{align*}
	这是一个上三角阵. \qed
\end{proof}

\section{Cayley-Hamilton定理}

Cayley-Hamilton定理是高等代数中特别重要的内容,我们介绍它的基本应用以及可以巧妙的解决很多问题.

\begin{theorem}[Cayley-Hamilton定理]
	设$A$是数域$\mathbb{K}$上的$n$阶方阵,$f(x)$是$A$的特征多项式,则$f(A)=O$.
\end{theorem}

\begin{proof}
	由Schur引理可知,矩阵$A$必定复相似一个上三角阵,即存在可逆阵$P$,使得$P^{-1} A P=B$,$B$
	为上三角阵,特征多项式是相似关系下的不变量,因此$A$和$B$具有相同的特征多项式. 则$f(B)=O$.
	\begin{align*}
		f(A) &= f(P B P^{-1} ) \\
		&= P f(B) P^{-1}
		\\
		&= O.
	\end{align*}  \qed{}
\end{proof}


在介绍应用之前,我们先介绍一个引理. 虽然不放在专门的一个章节,但也是很重要的内容,对于$AB=BA$这一类矩阵
,拿到这样的条件我们称为可交换性。但如果不是这样的条件,而是$AX=XB$,出来的却是$B$,可以把这样的
性质称为亚交换性,就和正定性和亚正定性一样。

\begin{proposition}
	设$A,B,C$分别是$m\times m, n\times n, m \times n$的矩阵,已知$AC=CB$,$C$的秩等于$r$.
	求证:$A$和$B$至少有$r$个相同的特征值.
\end{proposition}

\begin{proof}
	从特殊到一般.

	\emph{Step 1}:假设$C$为$C=\begin{pmatrix}
		I_r & O \\
		O & O
	\end{pmatrix}, A=\begin{pmatrix}
		A_{11} & A_{12} \\
		A_{21} & A_{22}
	\end{pmatrix}, B=\begin{pmatrix}
		B_{11} & B_{12} \\
		B_{21} & B_{22}
	\end{pmatrix}$,
	则
	\begin{align*}
	AC = \begin{pmatrix}
		A_{11} & O \\
		A_{21} & O
	\end{pmatrix}, \quad CB = \begin{pmatrix}
		B_{11} & B_{21} \\
		O & O
	\end{pmatrix}
	\end{align*}
	因为$AC=CB$,于是$A_{11} = B_{11}$,两个$r$阶矩阵一模一样,所以至少有$r$个相同的特征值.

	\emph{Step 2}:假设$C$为一般的矩阵,则一定存在可逆阵$P,Q$,使得$PCQ = \begin{pmatrix}
		I_r & O \\
		O & O
	\end{pmatrix}$.
	则
	\begin{align*}
		AC = A P^{-1} \begin{pmatrix}
			I_r & O \\
			O & O
		\end{pmatrix} Q^{-1}, \quad 
		CB = P^{-1} \begin{pmatrix}
			I_r & O \\
			O & O
		\end{pmatrix} Q^{-1} B
	\end{align*}
	\begin{align*}
		\bigl( P A P^{-1} \bigr) \begin{pmatrix}
			I_r & O \\
			O & O
		\end{pmatrix} = \begin{pmatrix}
			I_r & O \\
			O & O
		\end{pmatrix} \bigl( Q^{-1} B Q \bigr)
	\end{align*}
	令$A^\ast = P A P^{-1}, B^\ast = Q^{-1} B Q$,则是我们上面证明过的结论。因此,$A^\ast$和$B^\ast$
	至少有$r$个相同的特征值,因此$A$和$B$有至少$r$个不同的特征值. \qed{}
\end{proof}

\begin{proposition}
	设$A$为$m$阶方阵,$B$为$n$阶方阵,求证:若$A,B$无公共特征值,则矩阵方程$AX=XB$只有零解.
\end{proposition}

\begin{proof}
	\emph{Methods\,\,\uppercase\expandafter{\romannumeral 1}: Cayley-Hamilton}

	设$A$的特征多项式为$f(\lambda) = | \lambda I - A | $,则
	$$
		O = f(A)X = Xf(B)
	$$
	我们需要证明$f(B)$可逆即可左右两边同时消去.

	设$B$的所有特征值为$\mu_1, \mu_2, \dots, \mu_s$,则$f(B)$的所有特征值为$f(\mu_1),f(\mu_2),\dots,f(\mu_s)$,
	且$A,B$无公共的特征值,则$f(\mu_1) \neq 0, f(\mu_2) \neq 0, \dots, f(\mu_s) \neq 0$,则$f(B)\neq 0$,$f(B)$是可逆的,因此$X=0$. 
	
	\emph{Methods\,\,\uppercase\expandafter{\romannumeral 2}:}
	用到我们上面的命题.

	假设$C$为矩阵方程$AX=XB$其中一个解,若$C \neq 0$,则$r(C) \geq 1$,由命题可知,$A$和$B$
	至少有$r$个相同的特征值,与题目中没有公共的特征值矛盾,因此$C=0$,即这个矩阵方程只有零解.
	\qed{}
\end{proof}

下面是这个重要结论的应用,如果不知道的话,那做起来会很麻烦.

\begin{ex}{CMC}
	设$A$为$n$阶方阵,其$n$个特征值皆为偶数,试证明关于$X$的矩阵方程:$X+AX-XA^2=0$只有零解.
\end{ex}

\begin{proof}
	化简$(A+I)X=XA^2$,则若$A+I$和$A^2$无公共特征值即可.

	$A$的特征值皆为偶数,因此$A+I$的特征值都是奇数,$A^2$的特征值都是偶数(上三角化即可),因此无公共的
	特征值,则这个矩阵方程只有零解. \qed{}
\end{proof}

\begin{proposition}
	设$A$为$m$阶方阵,$B$为$n$阶方阵,证明:矩阵方程$AX-XB=C$有唯一解.
\end{proposition}

\begin{proof}
	构造一个同构映射:
	\begin{align*}
		\varphi: \,\, M_{m \times n}(\mathbb{K}) &\to M_{m \times n}(\mathbb{K}) \\
			X \quad \quad  &\mapsto AX-XB
	\end{align*}
	$AX-XB=O \Longleftrightarrow X=O$,即$\Ker \, \varphi =O$,是单射,且映射是线性同构,因此
	该线性映射是满射,所以存在唯一解$C$. 
	\qed{}

\end{proof}

\section{可交换性}

两个矩阵有公共的特征向量与矩阵乘法的可交换性有着密切的联系,这个问题和同时上三角化(对角化)有着类似的
思想. 不变子空间是解决这类问题的常用方法,我们通过题目来熟悉并掌握它.


$AB$可交换的一些简单的结论:
\begin{enumerate}
	\item $AB$可交换$\implies $ 至少有一个公共的特征向量.
	\item $AB$可交换$\implies $ 至少有一个公共的特征向量 $\implies $  $A,B$可同时化为上三角.
	\item $AB$可交换,且$A,B$都可对角化$\implies A,B$可同时对角化.
\end{enumerate}

我们先证明重要的引理.

\begin{lemma}[限制的可对角化]
	设$\varphi \in \mathscr{L}(V_{\mathbb{C}}^n)$,且$V_0$是$\varphi$-不变子空间,若$\varphi$
	可对角化,则$\varphi \lm{V_0}$也可对角化.
\end{lemma}

\begin{proof}
	设$\varphi$的极小多项式为$m(\lambda)$,$\varphi \lm{V_0}$的极小多项式为$g(\lambda)$.
	$$
		m(\varphi \lm{V_0})=m(\varphi)\lm{V_0}=0
	$$
	则$g(\lambda) | m(\lambda)$,且$\varphi$可对角化$\implies $$m(\lambda)$无重根.

	因此$g(\lambda)$无重根$\implies $$\varphi \lm{V_0}$也可对角化. \qed
\end{proof}

\begin{remark}
	我们可以容易理解到,只要$\varphi$可对角化,则限制在任意一个它的不变子空间上也可以对角化.
\end{remark}

\begin{lemma}[可交换的不变子空间]
	设$\varphi, \psi \in \mathscr{L}(V_{\mathbb{C}}^n)$,且$\varphi, \psi$可交换,即$\varphi \psi = \psi \varphi$.
	设$\varphi$的不同特征值为$\lambda_1,\lambda_2,\dots,\lambda_s$,$V_i$是特征值为$\lambda_i$的
	特征子空间,证明:$V_i$是$\psi$-不变子空间.
\end{lemma}

\begin{proof}
	对于任意的$x \in V_i$,我们有
	$$
		\varphi \bigl(\psi (x) \bigr)= \psi \varphi (x) = \psi (\lambda_i x) = \lambda_i \cdot \psi(x)
	$$
	则$\psi(x) \in V_i$,因此$V_i$是$\psi$-不变子空间. \qed{}
\end{proof}

\begin{remark}
	这个引理则告诉我们,对于可交换的线性变换,则任意一个的特征子空间一定是这两个线性变换的不变子空间.
\end{remark}

\begin{proposition}[公共特征向量]
	设$\phi, \varphi$是复线性空间$V$上的可交换的线性变换,即$\varphi \psi= \psi \varphi $,求证:

	$\varphi, \psi$至少有一个公共的特征向量.
\end{proposition}

\begin{proof}
	由代数学基本定理可知,维数大于零的复线性空间至少有一个特征值和特征向量,任取$\varphi$一个特征值$\lambda_0$和它
	对应的特征向量$\alpha_0$,设$V_0$是属于特征值$\lambda_0$的特征子空间,且是$\varphi$-不变子空间

	$\forall x \in V_0$,有
	$$
		\varphi \bigl(\psi (x) \bigr)= \psi \varphi (x) = \psi (\lambda_i x) = \lambda_i \cdot \psi(x)
	$$
	则$\psi(x) \in V_0$,所以$V_0$也是$\psi$-不变子空间,将$\psi$限制在$V_0$上,
	$V_0$是维数大于零的复线性空间,故$\psi \lm{V_0}$至少有一个特征值$\mu_0$和对应的特征向量$\alpha_0$,
	因此$\alpha_0$是$\varphi, \psi$共有的特征向量. \qed
\end{proof}

\begin{proposition}[同时上三角化]
	若$A,B$都是数域$\mathbb{K} $上的$n$阶方阵,且$AB=BA$,假定$A,B$的特征值都在数域$\mathbb{K}$中,
	求证:存在$\mathbb{K} ^n$上的可逆阵$P$,使得$P^{-1} A P,P^{-1} B P$都是上三角阵. 
\end{proposition}

\begin{proof}
	对矩阵的阶数$n$进行归纳,当$n=1$时,显然成立

	假设小于$n$时成立,下面证明等于$n$时也成立.

	由上面的命题可知,当矩阵$AB$可交换,则至少存在一组公共的特征向量,设一组公共的特征向量为$e$,对应的特征值
	为$\lambda_1,\mu_1$,于是有
	$$
		Ae = \lambda_1 e, \quad Be = \mu_1 e.
	$$
	将这个特征向量扩充为$\mathbb{K}^n$中的一组向量$\{ e_1, e_2,\dots, e_n \}$,把他们拼成
	形式行向量的形式,记作$P=(e_1, e_2, \dots, e_n)$,满足
	$$
		A P = (e_1, e_2, \dots, e_n) \begin{pmatrix}
			\lambda_1 & \ast \\
			& A_1
		\end{pmatrix}, \quad BP = (e_1, e_2, \dots, e_n) \begin{pmatrix}
			\mu_1 & \ast \\
			& B_1
		\end{pmatrix}
	$$
	注意到$P$为非异阵,则$P^{-1} A P =\begin{pmatrix}
		\lambda_1 & \ast \\
			& A_1
		\end{pmatrix}, P^{-1} B P = \begin{pmatrix}
			\mu_1 & \ast \\
			& B_1
		\end{pmatrix}$.
	
	因为$AB=BA$,我们需要$A_1B_1 = B_1 A_1$,通过简单验证可知,成立:
	\begin{align*}
		AB &= P^{-1} \cdot \begin{pmatrix}
			\lambda_1 & \ast \\
			& A_1
		\end{pmatrix} \bigl( P \cdot P^{-1} \bigr) \cdot
		\begin{pmatrix}
			\mu_1 & \ast \\
			& B_1
		\end{pmatrix} \cdot P = P^{-1} \cdot \begin{pmatrix}
			\lambda_1 & \ast \\
			& A_1
		\end{pmatrix}
		\cdot 
		\begin{pmatrix}
			\mu_1 & \ast \\
			& B_1
		\end{pmatrix} \cdot P \\
		BA &= P^{-1} \cdot \begin{pmatrix}
			\mu_1 & \ast \\
			& B_1
		\end{pmatrix} \bigl( P \cdot P^{-1} \bigr) \cdot
		\begin{pmatrix}
			\lambda_1 & \ast \\
			& A_1
		\end{pmatrix} \cdot P = P^{-1} \cdot
		\begin{pmatrix}
			\mu_1 & \ast \\
			& B_1
		\end{pmatrix}
		\cdot 
		\begin{pmatrix}
			\lambda_1 & \ast \\
			& A_1
		\end{pmatrix}
		\cdot P
		\\
		AB &= BA \implies 
		\begin{pmatrix}
			\lambda_1 \mu_1 & \ast \\
			& A_1B_1
		\end{pmatrix}
		=
		\begin{pmatrix}
			 \mu_1 \lambda_1 & \ast \\
			& B_1A_1
		\end{pmatrix}  
	\end{align*}
	则$A_1 B_1=B_1 A_1$,由归纳假设可知$A_1, B_1$可同时上三角化,即存在可逆阵$Q$. 令$R=P \begin{pmatrix}
		1 & O \\
		O & Q
	\end{pmatrix}$.
	\begin{align*}
		R^{-1}AR &= \begin{pmatrix}
			\lambda_1 & \ast \\
			& Q^{-1}A_1Q
		\end{pmatrix} \\
		R^{-1}BR &= \begin{pmatrix}
			\mu_1 & \ast \\
			& Q^{-1}B_1Q
		\end{pmatrix}
	\end{align*}
	\qed{}
\end{proof}

\begin{proposition}[同时对角化]
	设$A,B$是$n$阶矩阵,若$A$有$n$个不同的特征值,且$AB=BA$,求证:$B$相似于对角矩阵.
\end{proposition}

\begin{proof}
	\emph{Methods\,\,\uppercase\expandafter{\romannumeral 1}: 代数}

	$A$有$n$个不同的特征值,存在可逆阵$P$,使得$P^{-1}AP=diag\{ \lambda_1, \lambda_2,\dots
	,\lambda_n \}$,其中$\lambda_1, \lambda_2,\dots,\lambda_n$是$A$的特征值,根据$AB=BA$可得:
	\begin{align*}
		P \cdot diag\{ \lambda_1, \lambda_2,\dots,\lambda_n \} \cdot P^{-1} B &= 
		B \cdot P \cdot diag\{ \lambda_1, \lambda_2,\dots,\lambda_n \} \cdot P^{-1} \\
		diag\{ \lambda_1, \lambda_2,\dots,\lambda_n \} P^{-1} \cdot B P &= 
		P^{-1} B  P \cdot diag\{ \lambda_1, \lambda_2,\dots,\lambda_n \}
	\end{align*}
	则我们得到$P^{-1} B P$与$diag\{ \lambda_1, \lambda_2,\dots,\lambda_n \}$可交换,于是不妨设$A$是
	$diag\{ \lambda_1, \lambda_2,\dots,\lambda_n \}$,
	\begin{align*}
		diag\{ \lambda_1, \lambda_2,\dots,\lambda_n \} \cdot B = B \cdot diag\{ \lambda_1, \lambda_2,\dots,\lambda_n \}
	\end{align*}
	对比元素可知,$B$一定是一个对角阵,于是,可知,$P^{-1} B P$为对角阵,即相同的$P$使得他们
	同时对角化. 
	\qed{}
\end{proof}

\section{对角化1:极小多项式无重根的应用}

\begin{proposition}[极小多项式是相似关系的不变量]
	$A,B$都是$n$阶方阵,且$A$相似于$B$,则他们有相似的极小多项式.
\end{proposition}

\begin{proof}
	设$A$的极小多项式为$m_A(\lambda)$,$B$的极小多项式为$m_B(\lambda)$. 且$A$与$B$相似,
	则存在可逆阵$P$,使得$A=P^{-1} B P$
	\begin{align*}
		m_A(A) &= m_A(P^{-1} B P) = P^{-1} \, m_A (B) \, P = 0 
		\\
		&\implies m_B | m_A \\
		m_B(B) &= m_B(P A P^{-1}) = P \,m_B(A)\, P^{-1} = 0 
		\\
		&\implies m_A | m_B
	\end{align*}
	因此$m_A=m_B$.
	\qed{}
\end{proof}

\begin{proposition}
	设$A$适合$A^2=I_n$,则$A$必可对角化.
\end{proposition}

我们给出两种证法,第二种为极小多项式证法,之所以写第二条,来对比体现出其简洁性。

\begin{proof}
	\emph{Methods\,\,\uppercase\expandafter{\romannumeral 1}:}

	由已知我们可以得到$A$适合多项式$f(x)=x^2-1=(x-1)(x+1)$,则$A$的任意特征值$\lambda$也适合
	该多项式,解得$\lambda=1,\lambda=-1$(不计重数). 则$\lambda=1$时,特征子空间为$V_1=\Ker(A-I_V)$,
	当$\lambda=-1$时,特征子空间为$V_2=\Ker (A + I_V)$.

	下面证明一个引理:设$f_1(x),f_2(x)$为数域$\mathbb{R}$上的互素多项式,$\varphi$为$n$维线性空间
	$V$上的线性变换,且满足$f_1(\varphi) f_2(\varphi)=0$,则$V = V_1 \oplus V_2$,$V_1 = \Ker f_1(\phi),
	V_2 = \Ker f_(\varphi)$.

	$f_1(x), f_2(x)$互素$\implies $ $\exists u(x),v(x) \in R[x]$, \emph{s.t.},
	$$
	    f_1(x) \cdot u(x) + f_2(x) \cdot v(x) = 1, \text{令$x=\varphi$可得}
	$$
	$$
	f_1(\varphi) \cdot u(\varphi) + f_2(\varphi) \cdot v(\varphi) = I_V  \quad (\ast )
	$$
	$\forall \alpha \in V$,左右同时作用$\alpha$,
	$$
	f_1(\varphi) \cdot u(\varphi)(\alpha) + f_2(\varphi) \cdot v(\varphi)(\alpha) = \alpha,
	$$
	注意到$f_1(\varphi) u(\varphi) \in \Ker \, f_2(\varphi)$,$f_2(\varphi) v(\varphi) \in
	\Ker \, f_1(\varphi)$,且$\alpha$是任意的,则
	$$
		V = V_1 + V_2
	$$
	任取$\beta \in \Ker \, f_1(\varphi) \bigcap f_2(\varphi)$,两边同时作用$\beta$在$\ast $中.
	$$
		u(\varphi) \cdot f_1(\varphi) (\beta) + v(\varphi) f_2(\varphi) (\beta) = \beta
	$$
	$$
		0 + 0 = \beta,
	$$
	因为$\beta$是任取的,所以$\Ker \, f_1(\varphi) \bigcap \Ker \, f_2(\varphi)=0$,
	则$V=V_1 \oplus  V_2$.

	我们注意到,$f(x)=x^2-1=h(x) \cdot g(x), h(x)=x-1, g(x)=x+1$,且$\bigl( h(x), g(x) \bigr)=1$
	,我们把矩阵$A$视作线性变换,即可得$g(A)h(A)=0$,就是我们证明的引理。 因此全空间可以分解为:
	$$
		V = V_1 \oplus V_2
	$$
	即所有的特征子空间的直和为全空间$\implies $ $A$可对角化.


	\emph{Methods\,\,\uppercase\expandafter{\romannumeral 2}:}

	由已知我们可以得到$A$适合多项式$f(x)=x^2-1=(x-1)(x+1)$,则$A$的任意特征值$\lambda$也适合
	该多项式,即$f(\lambda)=\lambda^2-1=0$,设$A$的极小多项式为$m(\lambda)$,由极小多项式的
	性质可得:$m(\lambda) | f(\lambda)$,且$f(\lambda)$无重根,则$m(\lambda)$无重根,则根据
	极小多项式无重根可知$A$必可对角化.
	\qed
\end{proof}

\begin{ex}[ZJU 2007]
	设$n$阶方阵$A$满足$A^2-5A+6E=0$,其中$E$是单位矩阵,证明:

	\vspace{0.2cm}
	$A$相似于对角矩阵. 如果$A$的行列式
	等于$2^m3^{n-m} (0<m<n),m\text{是正整数}$. 求与$A$相似的一个对角阵.
\end{ex}

\begin{proof}
	由上面一题的经验,我们可以直接秒杀这道题,矩阵$A$适合多项式$f(x)=x^2-5x+6$,则特征值也适合
	这个多项式,$f(\lambda)=(\lambda-2)(\lambda - 3)$.

	设矩阵$A$的极小多项式为$m(x)$,则$m(\lambda) | f(\lambda)$,易得多项式$f(\lambda)$
	无重根,则$A$的极小多项式$m(x)$也没有重根,可以推出$A$可对角化.

	由前面的引理可知,行列式是相似关系下的不变量,存在可逆矩阵$P$使得:
	$$
		P^{-1} A P = B
	$$
	$B$是对角阵,则$|A|=|B|$,因此其中一个相似的对角阵为:
	$
	\begin{pmatrix}
		2 & & \\
		& \ddots & \\
		& & 3
	\end{pmatrix}
	$,其中$m$个$2$,$n-m$个$3$.
	\qed
\end{proof}

\begin{proposition}
	设$A$为$m$阶矩阵,$B$为$n$阶矩阵,$C$为$m \times n$矩阵,$M=\begin{pmatrix}
		A & C \\
		O & B
	\end{pmatrix}$,求证:若$M$可对角化,则$A,B$均可对角化.
\end{proposition}

两种证法,白皮书上p295用完全特征向量系较为繁琐,这里用极小多项式更简洁。

\begin{proof}
	设$M$的极小多项式为$m(\lambda)$,则
	$$
		0 = m(M) = \begin{pmatrix}
			m(A) & \ast \\
			O & m(B)
		\end{pmatrix} 
	$$
	从而$m(A)=0,m(B)=0$,则利用极小多项式的性质:极小多项式整除任何矩阵适合的多项式.
	$$
		m_A(\lambda) | m(\lambda)  \quad m_B(\lambda) | m(\lambda)
	$$
	再由对角化的充要条件,$M$可对角化,则$M$的极小多项式$m(\lambda)$无重根,$m_A(\lambda),m_B(\lambda)$也
	无重根$\implies A,B$均可对角化. \qed
\end{proof}

\chapter{二次型与内积空间}

\begin{introduction}
	\item 正定型与正定阵
	\item 实对称矩阵的正交相似标准型
	\item 同时合同三角化
	\item 实正规阵的正交相似标准型
\end{introduction}

\section{基本定理梳理}

\section{同时合同对角化}

我们在可交换性那边,介绍过可同时上三角化,对角化,可他们的要求都是需要交换性,在实对称这边,我们
还有类似可交换性那边一样的结论,相当重要,需要像定理一样铭记在心.

\begin{lemma}[同时合同对角化]
	设$A$是$n$阶正定实对称阵,$B$是同阶实对称阵,求证:必存在可逆阵$C$使得
	$$
		C^{\prime} A C = I_n,\quad C^{\prime} B C = diag\{ \lambda_1, \lambda_2, \dots, \lambda_n \}
	$$
	其中,$\{ \lambda_1, \lambda_2, \dots, \lambda_n \}$是矩阵$A^{-1}B$的特征值.
\end{lemma}

\begin{proof}
	$A$是正定实对称,则存在可逆阵$P$,使得$P^{\prime} A P=I_n$,根据前面的引理可以知道$P^{\prime} B P$也为
	实对称阵,则存在正交阵$Q$,使得$Q^{\prime} P B P Q=diag\{\lambda_1, \lambda_2,\dots, \lambda_n\}$.

	令$C=PQ$,那么
	$$
		C^\prime (\lambda A - B) C  = \lambda I_n - C^\prime B C = diag\{ \lambda - \lambda_1, \lambda - \lambda_2,\dots, \lambda - \lambda_n \}  
	$$
	因此$\lambda_i$是$|\lambda A - B |$的根,$A$可逆,则$\lambda_i$也是$|\lambda I_n - A^{-1}B|$的根.
	\qed{}
\end{proof}

\begin{proposition}
	设$A$是$n$阶实对称矩阵,$B$是同阶半正定阵,求证:
	$$
		| A + B | \geq | A | + | B |
	$$
	等号成立的充要条件为$B=O$.
\end{proposition}

\begin{proof}
	由上面的引理可知,存在可逆阵$P$,使得$P^{\prime} A P=I_n, P^{\prime} B P= \{ \lambda_1, \lambda_2,\dots, \lambda_n \}$,
	其中$\lambda_1, \lambda_2, \dots, \lambda_n$为$A^{-1}B$的特征值. 由$B$是半正定阵可知,$\lambda_i \geq 0$.

	\begin{align*}
		|P^\prime | |(A+b)| |P| = |P^{\prime}(A + B)P| = | P^\prime A P + P^\prime B P | &= 
		| I_n + diag\{ \lambda_1, \lambda_2, \dots, \lambda_n \} | \\
		&= (1+\lambda_1)(1+\lambda_2)\cdots (1+\lambda_n) \\
		&\geq 1 + \lambda_1 \lambda_2 \dots \lambda_n \\
		&= |P^\prime|(| A | + | B |) |P| \\
	\end{align*}
	\qed{}
\end{proof}

\begin{ex}[电子科大 2021]
	已知矩阵$A,B$均为$n$阶正定阵.
	\begin{enumerate}
		\item 证明:$A+2021B$也为正定阵.
		\item 证明:$| A+2021B | > | A |$.
	\end{enumerate}
\end{ex}

\begin{proof}
	(1). $A,B$均为正定阵...很好证明,我们用正定阵的性质,对于任意的$\alpha \in \mathbb{R}^n \neq 0$,只要证明左乘
	右乘恒大零即可.
	$$
		\alpha^\prime (A + 2021B) \alpha = \alpha^\prime A \alpha + 2021 \alpha^\prime B \alpha > 0.
	$$

	(2). 用刚才的引理,存在可逆阵$P$,使得$P^{\prime} A P=I_n, P^{\prime} B P= \{ \lambda_1, \lambda_2,\dots, \lambda_n \}$,
	其中$\lambda_1, \lambda_2, \dots, \lambda_n$为$A^{-1}B$的特征值. 且$A^{-1}B$仍为正定阵,因此这些
	特征值全部都大于0.
	\begin{align*}
		|P^\prime | | (A+2021B) | | P | = | P^\prime (A + 2021B) P | = | P^\prime A P + 2021 P ^\prime B P| &= | I_n + 2021 diag\{ \lambda_1, \lambda_2, \dots, \lambda_n \} | \\
		&= (1+2021\lambda_1)(1+2021\lambda_2)\cdots (1+2021\lambda_n) \\
		&> 1 \\
		&= | P^\prime A P| \\
		&= |A|
	\end{align*}
	\qed{}
\end{proof}

\section{同时合同标准化(正交标准型)}

正规阵有着正交相似标准型,特别地,在反对称阵和正交阵使用的较多。我们介绍重要的同时合同标准化.

\begin{itemize}
	\item 正定实对称阵 + 实对称阵 $\implies$ 同时合同对角化.
	\item 正定实对称阵 + 实反对称阵 $\implies$ 同时合同标准化(正交标准型).
\end{itemize}

\begin{lemma}[同时合同标准化]
	设$A$为$n$阶正定实对称阵,$S$实同阶实反对称阵,则存在可逆阵$C$,使得
	$$
		C^\prime A C = I_n, \, C^\prime B C = diag\left\{ \begin{pmatrix}
			0 & b_1 \\
			-b_1 & 0
		\end{pmatrix}, \dots, \begin{pmatrix}
			0 & b_r \\
			-b_r & 0
		\end{pmatrix},0,\dots, 0\right\},
	$$
	其中$b_1,\dots,b_r$为非零实数.
\end{lemma}

\begin{proof}
	$A$为正定阵,则存在可逆阵$P$,使得$P^\prime A P =I_n$,且$p^\prime B P$仍为实反对称阵,
	于是存在正交阵$Q$,使得$Q^\prime (P^\prime A P) Q=diag\left\{ \begin{pmatrix}
		0 & b_1 \\
		-b_1 & 0
	\end{pmatrix}, \dots, \begin{pmatrix}
		0 & b_r \\
		-b_r & 0
	\end{pmatrix},0,\dots, 0\right\}$为正交相似标准型,$b_1,\dots,b_r$为非零实数,且$Q^\prime (P^\prime A P)Q=I_n$.
	
	令$C=PQ$即可. 
	\qed{}
\end{proof}