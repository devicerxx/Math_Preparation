\chapter{谢启鸿每周一题}

\begin{introduction}
	\item zzz
\end{introduction}

每周一题有些好难QAQ... 

\section{2022 高代 \uppercase\expandafter{\romannumeral 2}}

\begin{proposition}
    设$A$为$n$阶实对称阵,$B$为$n$阶半正定实对称阵,满足$|A+iB|=0$,求证:存在$n$维非零
    实列向量,使得$A\alpha = B \alpha = 0$.
\end{proposition}

\begin{proof}
    这道题收到了某公众号的不良影响,记忆来记那个东西是真的害人,有些公众号写的错误自己都不
    检查一下吗???

    记录一下正经解法,看完觉得这个过程没有那么难构造,一步一步构造出来就好,缺什么补什么,谈不上巧妙这题,检验基础不错的题。

    看到$B$为半正定实对称阵,自带许多有用的性质,$A$相对比较普通。我们先尝试一下,将$iB$这个给变的
    简单一点。存在可逆阵$C$,使得$C^\prime B C = \begin{pmatrix}
        I_r & O \\ O & O
    \end{pmatrix}$,其中$r(B)=r$.
    $$
        |A+iB|=0 \implies |C^\prime| \cdot |A+iB| \cdot |C| = 0 \implies
        |C^\prime A C + i \cdot I_r|.
    $$
    这个时候$i$这边变简单了,注意到$\ts{C} A C $仍为对称阵,存在可逆(正交阵)$P$,使得
    $\ts{P} \ts{C} A C P=\{ \lambda_1, \lambda_2 , \dots, \lambda_n \}$,其中$\lambda_1,\lambda_2,\dots,\lambda_n$
    为$\ts{C}AC$的全体特征值,于是有:
    $$
        |P^\prime| | \ts{C} A C + i \cdot I_r| | P | = 0 \implies 
        |diag\{ \lambda_1, \lambda_2, \dots, \lambda_n \} + i \cdot I_r| = 0
    $$
    然后这个行列式可以拆解为$(\lambda_1 + i)(\lambda_2+ i) \cdots (\lambda_r + i) \lambda_{r+1} \cdots \lambda_n$,
    其中前$r$项是不为$0$的,于是存在正整数$k \in [r+1, n]$,$\lambda_{k}=0$,不妨设$k=n$。取$\alpha=
    CP(0,\dots,1)^\prime$,于是有:
    $$
        \ts{P} \ts{C} (A+iB) \alpha = diag\{ (\lambda_1+i),\dots,(\lambda_r+i),\lambda_{r+1},\dots,\lambda_n \}(0,\dots,1)^\prime 
        = 0
    $$
    其中$P,C$非异,因此消去,故$(A+iB)\alpha=0$,于是$A\alpha=0,B\alpha=0$.
    \qed
\end{proof}

